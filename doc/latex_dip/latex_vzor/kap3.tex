\chapter{Matematický model}

Tu je vhodné uviesť ďalej používané základné pojmy a tvrdenia. 
Čitetel ocení ak sú tieto demonštrované na výstižných ilustračných príkladoch. 
Opäť nezabudnúť dôsledne citovať autorov. 
U vlastných výsledov sa zse nehambiť na túto skutočnosť upozorniť -- napríklad názvom podkapitoly.


\section{Problém pažravej trojhodnotovej stonožky}

Nech $m,n,a,b$ prirodzené čísla $n,m \ge 2$ a $0<a<b$. 
Sú dané $n$-tice prirodzených čísel ${\mathrm a}=(a_1,\dots,a_j,\dots,a_n)$
a ${\mathrm b}=(b_1,\dots,b_j,\dots,b_n)$ také, že $a_j+b_j \le m$.
Hľadá sa taká matica ${\mathrm A}=(a_{ij})_{m\times n}$, ktorá v každom
stĺpci $j$ obsahuje aspoň $a_j$ prvkov rovných $a$ a aspoň $b_j$ prvkov rovných $b$,
pričom rozdiel medzi najväčším a najmenším  riadkovým súčtom prvkov matice ${\mathrm A}$ 
je čo najmenší.

Označme $M=\{1,2,\dots,m\}$ a $N=\{1,2,\dots,n\}$. 
Uvažujme premenné matice 
${\mathrm X} = (x_{ij})_{m\times n}$, ${\mathrm Y}=(y_{ij})_{m\times n}$ s prvkami

$$
x_{ij}=\left\{\begin{array}{ll}
	 1, &\mbox{ak } a_{ij}=a \\
	 0, &\text{inak}         \end{array}\right.
\quad 
y_{ij}=\Bigg\{\bigg\{\Big\{\big\{\begin{array}{ll}
	 1, &\mbox{ak } a_{ij}=b \\
	 0, &\mbox{inak}.        \end{array}
\quad 
y_{ij}=\Big\{\begin{array}{ll}
	 1, &\mbox{ak } a_{ij}=b \\[-.5em]
	 0, &\mbox{inak}.        \end{array}
$$

Zátvorka \verb|\left\{| musí mať pravý pár,  napr. \verb|\right\}|  alebo,  ak nechceme mať na
druhej strane nič, použijeme príkaz s bodkou \verb|\right.|

Nech $m,n,a,b$ prirodzené čísla $n,m \ge 2$ a $0<a<b$. 
Sú dané $n$-tice prirodzených čísel ${\mathrm a}=(a_1,\dots,a_j,\dots,a_n)$
a ${\mathrm b}=(b_1,\dots,b_j,\dots,b_n)$ také, že $a_j+b_j \le m$.
Hľadá sa taká matica ${\mathrm A}=(a_{ij})_{m\times n}$, ktorá v každom
stĺpci $j$ obsahuje aspoň $a_j$ prvkov rovných $a$ a aspoň $b_j$ prvkov rovných $b$,
pričom rozdiel medzi najväčším a najmenším  riadkovým súčtom prvkov matice ${\mathrm A}$ 
je čo najmenší.



Problém možno riešiť ako nasledujúcu úlohu matematického programovania:

\begin{alignat}{2} \label{u_1}
& z_2-z_1 \rightarrow \min             &                  &             \\
& \sum_{i\in M} x_{ij} \ge a_j,        &j\in N,           & \label{u_2} \\
& \sum_{i\in M} y_{ij} \ge b_j,        &j\in N,           & \label{u_3} \\
& x_{ij} + y_{ij} \le 1,               &i\in M, j\in N,   & \label{u_4} \\
& \sum_{i\in M} x_{ij}+ y_{ij} \le m,  &j\in N,           & \label{u_5} \\
& z_1 \le \sum_{j\in N} a\cdot x_{ij}+b\cdot y_{ij} \le z_2,  
                                       &i\in M,           & \label{u_6} \\
& x_{ij}, y_{ij} \in \{0,1\},          &i\in M, j\in N,   & \label{u_7} \\
& z_1, z_2 \ge 0.                      &                  & \label{u_8}
\end{alignat}

alebo ináč usporiadané a niektoré sumy ináč zobrazené:

\begin{alignat}{2} \label{u_1}
& z_2-z_1 \rightarrow \min             &&                               \allowdisplaybreaks\\
& \sum_{i\in M} x_{ij} \ge a_j,        &&j\in N,            \label{v_2} \allowdisplaybreaks\\
&\textstyle \sum_{i\in M} x_{ij} \ge a_j,          
                                       &&j\in N,            \nonumber   \allowdisplaybreaks\\
&\textstyle \sum\limits_{i\in M} x_{ij} \ge a_j,
                                       &&j\in N,            \nonumber   \allowdisplaybreaks\\
& \sum_{i\in M} y_{ij} \ge b_j,        &&j\in N,            \label{v_3} \allowdisplaybreaks\\
& x_{ij} + y_{ij} \le 1,               &&i\in M, j\in N,    \label{v_4} \allowdisplaybreaks\\
& \sum_{i\in M} x_{ij}+ y_{ij} \le m,  &&j\in N,            \label{v_5} \allowdisplaybreaks\\
z_1 \le &\sum_{j\in N} a\cdot x_{ij}+b\cdot y_{ij} \le z_2,\qquad  
                                       &&i\in M,            \label{v_6} \allowdisplaybreaks\\
& x_{ij}, y_{ij} \in \{0,1\},          &&i\in M, j\in N,    \label{v_7} \allowdisplaybreaks\\
& z_1, z_2 \ge 0.                      &&                   \label{v_8}
\end{alignat}



Jednotkové hodnoty premenných $x_{ij}$ resp. $y_{ij}$ zodpovedajú umiestneniu
hodnoty $a$ resp. $b$ v $i$-tom riadku a v $j$-tom stĺpci hľadanej matice.
Optimálnym riešením je potom matica ${\mathrm A} = (a_{ij})_{m\times n}$  s prvkami
%
$$ a_{ij} = a\cdot x_{ij}+b\cdot y_{ij}.$$

V cieľovej funkcii~\eqref{u_1} je hodnota premennej $z_2$ rovná najväčšiemu
riadkovému súčtu prvkov matice a hodnota premennej $z_1$ zas najmenšiemu
riadkovému súčtu. Podmienky~\eqref{u_2} a~\eqref{u_3} zabezpečujú, že bude
vybraných najmenej $a_j$ hodnôt $a$ a najmenej $b_j$ hodnôt $b$ v každom
stĺpci $j$. 
Podmienka~\eqref{u_4} zabráni umiestneniu oboch nenulových hodnôt
$a,b$ do jediného prvku $a_{ij}$ matice. 
Podmienka~\eqref{u_5}
obmedzuje zhora celkový počet nenulových prvkov v každom stĺpci počtom $m$
-- riadkov matice. Podmienkou~\eqref{u_6}
sú definované horná $z_2$ a dolná $z_1$ hranica riadkových súčtov. 
Obmedzenia premených~\eqref{u_7} a~\eqref{u_8} sú obligatorné.



