\chapter*{Úvod}
\addcontentsline{toc}{chapter}{Úvod}


Hviezdičková konvencia funguje ako normálna kapitola,  ale nie je číslovaná 
a nezobrazuje sa v obsahu.
To znamená,  že sa číslovanie nasledujúcich kapitol posunie ---  to isté sa stane aj 
s~obsahom --- ale súbor treba preložiť minimálne dva razy.
Aby sa kapitola (sekcia, \dots) zobrazila do obsahu je potrebné zadať príkaz:

\verb|\addcontentsline{toc}{chapter}{Úvod}|

resp.

\verb|\addcontentsline{toc}{section}{Sekcia}|


Súbor treba prekladať pomocou \verb|pdflatex praca|  \fbox{pdflatex praca}.

E-mail adresa a WWW stránka (interaktívne) sa píšu:

\verb|\href{mailto:beerb@frcatel.fri.uniza.sk}{beerb@frcatel.fri.uniza.sk}|

\verb|\url{http://frcatel.fri.uniza.sk/~beerb}|

a výsledok je:

\href{mailto:beerb@frcatel.fri.uniza.sk}{beerb@frcatel.fri.uniza.sk}

\url{http://frcatel.fri.uniza.sk/~beerb}




Tu by bolo dobré zoznámiť a zaradiť problematiku práce. 
Je dobré mať na pamäti, na základe akých kriterii bude oponent hodnotit túto prácu. 
Náročnosť zadania sa hodnotí slovne ako  malá, stredná, veľká na základe nasledujúcich kritérii:

\begin{itemize}

\item[$\heartsuit$]{teoretické znalosti,}

\item[$\ast$]{invenčnosť, tvorivosť,}

\item{experimenálna činnosť}

\item{technické práce vrátane programovania,}

\item{návrh algoritmu, datových štruktúr,}

\item{informačno rešeržný prieskum a syntéza.}

\end{itemize}

Dôležitejšie pre záverečné hodnotenie je však je bodové hodnotenie 
na základe nasledujúcich kritérii:

\begin{enumerate}

\item{Hĺbka analýzy vo vzťahu k téme [10b]}

\item[1b)]{Adekvátnosť použitých metód [15b]}

\item{Splnenie cieľov zadania [20b]}

\item{Kvalita riešenia [15b]}

\item{Logická stavba, nadvúznosť, úplnosť, zrozumiteľnosť [10b]}

\item{Formálna gramatická úroveň práce, dokumentácie a prezentácie 10b}

\end{enumerate}

Ak by oponent robil toto hodnotenie v Exelovskej tabuľke a v bunke $D21$ by mal
súčet bodov, potom výsledná známka bude

{\scriptsize
\begin{verbatim}
=if(D21>=90;"A";if(D21>=80;"B";if(D21>=70;"C";if(D21>=60;"D";if(D21>=50;"E";"Fx"))))).
\end{verbatim}
}


