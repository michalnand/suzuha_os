\begin{titlepage}
\phantom.

\bigskip

\begin{center}
{\sc\LARGE Žilinská Univerzita v Žiline}
\medskip

{\sc\Large Fakulta riadenia a informatiky}

\vfill\vfill\vfill\vfill

{\sc\LARGE Diplomová práca}

\medskip

{\large Študijný odbor: {\bf Počítačové inžinierstvo}}
\end{center}


\vfill\vfill\vfill\vfill


\phantom.\hfill
\begin{minipage}{10cm}
\begin{center}
{\large\bf Bc. Michal Chovanec}

\medskip

{\bf OPERAČNÝ SYSTÉM PRE JEDNOČIPOVÉ MIKROPOČÍTAČE S JADROM CORTEX-M3} 

\medskip

Vedúci: {\bf Ing. Ján Kapitulík, PhD.}

\medskip
 
\hfill
Reg.č. 39/2012
\hfill
Máj 2013
\hfill\phantom.
\end{center}
\end{minipage}
\hspace{1.7cm}\phantom.

\vspace{2.9cm}

\phantom.
\end{titlepage}



%vlozit zadanie prace
\newpage
zadanie prace

%--------------------------------------------------------------------------------------
%%% slovensky abstrakt

\begin{abstract}

\noindent
{\sc Chovanec Michal:} {\em OPERAČNÝ SYSTÉM PRE JEDNOČIPOVÉ MIKROPOČÍTAČE 
S JADROM CORTEX-M3 }
[Diplomová práca] 

\noindent
Žilinská Univerzita v~Žiline,  
Fakulta riadenia a informatiky,  
Katedra technickej kybernetiky.

\noindent  
Vedúci: Ing. Ján Kapitulík, PhD.
 
\noindent  
Stupeň odbornej kvalifikácie:
Inžinier v~odbore Počítačové inžinierstvo Žilina. 

\noindent
FRI ŽU v~Žiline, 2013

\bigskip

V práci je predstavená realizácia operačného systému pre jednočipové mikropočítače rodiny ARM Cortex-M3 a Cortex-M4. Celý systém bol napísaný a vyvinutý použitím open source nástrojov v prostredí Ubuntu Linux.
	Operačný systém realizuje preemptívny multitasking, správu kritických sekcií, systém posielania správ a základné knižnice pre spoluprácu s hardvérom. Výsledné riešenie bolo implementované a odladené na doske STM32 Discovery kit, vlastnej doske s STM32 a vývojovom kite Stellaris Launchpad.
	K jadru systému je pripojených niekoľko príkladov, ktoré demonštrujú funkcie systému a uľahčujú vývoj vlastnej aplikačnej časti.


\end{abstract}


%--------------------------------------------------------------------------------------
%%% anglicky abstrakt


\selectlanguage{english}
\begin{abstract}

\noindent
{\sc Chovanec Michal:} {\em Operation system for single-chip microcontrollers with Cortex-M3 core
}
[Diploma thesis] 

\noindent
University of Žilina,  
Faculty of Management Science and Informatics, 
Department of technical cybernetics.
 
\noindent
Tutor:  Ing. Ján Kapitulík, PhD.
 
\noindent
Qualification level:
Engineer in field Computer engineering Žilina: 

\noindent
FRI ŽU v Žiline, 2013

\bigskip

This work presents realization of operating system for single chip microcontrollers based on ARM Cortex-M3 and Cortex-M4 familly. Whole system has been written and developed using open source tools in Ubuntu Linux enviroment.
	Operating system realize preemptive multitasking, crytical sections management, message sending system and basic hardware libraries. Finally solution was implementated and tuned on STM32 Discovery kit, own board witch STM32 and development kit Stellaris Lauchpad.
	To system core is included five exeamples, which demonstatred system functionality and make easy to develop own application.

\end{abstract}
\selectlanguage{slovak}


%%%%%%%%%%%%%%%%%%%%%%%%%%%%%%%%%%%%%%%%%%%%%%%%%%%%%%%%%%%%%%%%%%%%%%%
\newpage

\centerline{\bf Prehlásenie}

\vspace{2em}

\noindent
Prehlasujem, že som túto prácu napísal samostatne a že som uviedol 
všetky použité pramene a literatúru, z ktorých som čerpal. 

\vspace{2em}

\noindent
V~Žiline, dňa 10.5.2013
\hfill
Michal Chovanec



%%%%%%%%%%%%%%%%%%%%%%%%%%%%%%%%%%%%%%%%%%%%%%%%%%%%%%%%%%%%%%%%%%%%%%%
\newpage
{\Large \bf{Predhovor}}


Operačný systém je tak bežný program v každom počítači, že mnoho užívateľov ani netuší, čo všetko za ním je. V práci som riešil analýzu a implementáciu jednoduchého operačného systému s podporou preemptívneho multitaskingu. Výsledkom je funkčný systém implementovateľný do jadier Cortex M3 a Cortex M4. Práca tak umožňuje každému na jednoduchom príklade vidieť fungovanie multitaskingu, semafórov, systému správ, či súborovému systému, prípadne použit systém ako podklad pre uľahčenie vývoja vlastnej aplikácie. Celý vývoj bol uskutočnený s využitím open source nástrojov, ako ukážka progresívnejšej cesty vývoja. Rovnako je celá práca k dispozícií ako open source.

Poďakovanie patrí môjmu vedúcemu práce Ing. Jánovi Kapitulíkovi, PhD. a vedúcemu katedry technickej kybernetiky, prof. Ing. Jurajovi Mičekovi, PhD. , za cenné rady pri vývoji. Veľká vďaka patrí aj Mgr. Michalovi Kaukičovi, CSc. za ukázanie možnosti vývoja s použitím oss nástrojov.
